\documentclass{article}
\usepackage[utf8]{inputenc}
\usepackage{amsmath}
\usepackage{graphicx}
\usepackage{BeginnerStyleFile}
\usepackage{scrextend}
\usepackage{ifthen}
\usepackage{listings}
\graphicspath{ {images/} }

\newcommand{\gfcb}[1]{%
    \fcolorbox{white}{gray!10!}{\quad\strut #1\quad}
    }
\newcommand{\cop}[1]{%
    \texttt{\detokenize{#1}}
    }

\title{Homework 2}
\author{Evan Hughes}
\date{January 2023}

\begin{document}
\maketitle
\begin{enumerate}

    %%%%%%%%%%%%%%%%%%%%
    \item Find the gcd of the given pair of numbers $(a, b)$ and express at least one of the gcd's as a $\mathbb{Z}$-linear combination of $a$ and $b$.
    %%%%%%%%%%%%%%%%%%%%
    
    \begin{enumerate}
        \item $(56,72)$
        
        $8 = (-5)56+(4)72$
        \item $(306,657)$
        
        $9$
        \item $(272,1479)$
        
        $17$
        \item $(1103,465)$

        $1$
    \end{enumerate}
    
    %%%%%%%%%%%%%%%%%%%%
    \item Let $p \in \mathbb{Z}$ be a prime integer.
    %%%%%%%%%%%%%%%%%%%%
    \begin{enumerate}
        \item Show that if $p > 3$, them $p$ is of the form $6k+1$ or $6k-1$ for some integer $k$.
        
        $p = 6k + n$ where $n$ is one of $0,1,2,3,4,5$. If n is 0, 2, or 4, then $p$ is even, so $p$ is not prime.
        If $n$ is 3, then $p$ is divisible by $3$, and not prime. This only leaves $1$ and $5$, $5 \\equiv -1 \pmod 6$, 
        therefore $p$ is of the form $6k-1$ or $6k+1$.
        
        \item If $p > 5$, show that dividing $p$ by $10$, can only leave remainders of $1, 3, 7,$ or $9$, and find examples of primes with these remainders.
        
        $p = 10k + n$ where $n$ is one of $0,1,2,3,4,5,6,7,8,9$. If $n$ is 0, 2, 4, 6, or 8, then $p$ is even, so $p$ is not prime.
        If remainder is $5$, then $p$ is divisible by $5$, and not prime. This only leaves $1, 3, 7,$ or $9$.

        $19 \pmod 10 = 9$

        $17 \pmod 10 = 7$

        $13 \pmod 10 = 3$

        $11 \pmod 10 = 1$

    \end{enumerate}

    %%%%%%%%%%%%%%%%%%%%
    \item Find the smallest positive integer in the given sets.
    %%%%%%%%%%%%%%%%%%%%
    \begin{enumerate}
        \item $\{ 6u + 15v \colon u,v\in\mathbb{Z} \}$
        
        $3 = (3)6 + (-1)15$
        
        $3 = $gcd$(6,15)$
        \item $\{ 12r + 17s \colon r,s\in\mathbb{Z} \}$
        
        $1 = (10)12 + (-7)17$

        $1 = $gcd$(12,17)$
    \end{enumerate}
    

    %%%%%%%%%%%%%%%%%%%%
    \item Let $a,b,c,d$ be integers.
    %%%%%%%%%%%%%%%%%%%%
    \begin{enumerate}
        \item If $a\mid c$ and $b\mid c$, is it necessary that $ab\mid c$? 
        
        If $a = 3$ and $b = 6$, and $c = 6$, then $a\mid c$ and $b\mid c$, but $ab$ does not divide $c$.

        \item Prove that if $a\mid c$ and $b\mid c$, and gcd$(a,b) = 1$, then $ab\mid c$.
        
        If gcd$(a,b) = 1$, then $1 = am + bn$ for some integers $m$ and $n$. Then $c = c(am) + (c)bn$.
        Because $a,b$ divide $c$, $c = as$ and $c = bt$ for some integers $s$ and $t$. Then $c = abns + abmt$. $c = ab(ns+mt)$, so $ab\mid c$.

        \item Prove that if gcd$(a,b) = d$, then $ab\mid cd$.
        
        If gcd$(a,b) = d$, then $d = am + bn$ for some integers $m$ and $n$. Then $cd = c(am+bn)$.
        $cd = cam + cbn$. Not sure where to continue this proof.
    \end{enumerate}

    %%%%%%%%%%%%%%%%%%%%
    \item Let $p$ be an integer other than $0$ or $\pm1$. Prove that if $p$ has the property
    \begin{equation*}
        \forall b,c\in\mathbb{Z}, p\mid bc \implies p\mid c \text{ or } p\mid b
    \end{equation*}
    then $p$ is a prime number.
    %%%%%%%%%%%%%%%%%%%%

    Since $p$ is prime if $-p$ is prime, we assume that
    $p > 1$. Suppose that $p = bc$ for some positive integers $b$ and $c$. Then
    $0 < b \leq p$ and $0 < c \leq p$. By the given properties, $p\mid b$ or $c$. Thus,
    $b = p$ and $c = 1$ or $c = p$ and $b = 1$. This shows that the only positive
    divisors of $p$ are $1$ and $p$. Therefore, the only divisors of $p$ are $\pm1$, $\pm p$;
    Therefore, $p$ is prime.



\end{enumerate}
    
\end{document}