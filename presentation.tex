\documentclass{article}
\usepackage[utf8]{inputenc}
\usepackage{amsmath}
\usepackage{graphicx}
\usepackage{BeginnerStyleFile}
\usepackage{scrextend}
\usepackage{ifthen}
\usepackage{listings}
\graphicspath{ {images/} }

\newcommand{\gfcb}[1]{%
    \fcolorbox{white}{gray!10!}{\quad\strut #1\quad}
    }
\newcommand{\cop}[1]{%
    \texttt{\detokenize{#1}}
    }

\title{Homework 3 problem 6, presentation}
\date{January 2023}

\begin{document}
\maketitle
\begin{enumerate}

    %%%%%%%%%%%%%%%%%%%%
    \item Prove the following statements.
    %%%%%%%%%%%%%%%%%%%%
    
    \begin{enumerate}
        \item If $a \in \mathbb{Z}_n$ is a unit, then $a$ is not a divisor of zero.
        
        Let $a$ be a unit of $\mathbb{Z}_n$, and assume $a$ is a divisor of zero. 
        
        Because $a$ is a unit there exists $b \in \mathbb{Z}_n$ such that $ab = 1$. 
        And because $a$ is a divisor of zero there exists
        $c \in \mathbb{Z}_n$ such that $ac = 0$ where $c \neq 0$.

        $[0] = [0 * b]$

        $0 = b(ac)$

        $0 = (ba)c$

        $0 = (1)c$

        $0 = c$

        Which is a contradiction. Therefore, $a$ is not a divisor of zero.

        \item If $a \in \mathbb{Z}_n$ is not a divisor of zero, then $a$ is a unit.
        
        Contrapositive of the previous statement. If $a$ is not a unit, then $a$ is a divisor of zero.

        If $a$ is not a unit of $\mathbb{Z}_n$, then gcd($a$, $n$) $= k \neq 1$.

        Then let $c = \frac{n}{k}$, $c \in \mathbb{Z}$, and so $0<c<n$ and $b \in \mathbb{Z}_n$.

        $ac = a(\frac{n}{k})= (\frac{a}{k})n$

        Since $k \mid a$, $ac$ is equal to an integer times $n$
        Therefore $ac = 0$. Meaning $a$ is a divisor of zero.

        Because the contrapositive is true, the original statement is true.
        

    \end{enumerate}

\end{enumerate}
    
\end{document}