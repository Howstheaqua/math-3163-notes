\documentclass{article}
\usepackage[utf8]{inputenc}
\usepackage{amsmath}
\usepackage{graphicx}
\usepackage{BeginnerStyleFile}
\graphicspath{ {images/} }

\title{1.2 Divisibility}
\author{Evan Hughes}
\date{January 2023}

\begin{document}
\maketitle
\subsection*{Definition of Divisibility}

Let $a$ and $b$ be integers with $b \neq 0$. 
We say that $b$ divides $a$( or that $b$ is a divisor of $a$, or that $b$ is a factor of $a$) if $a=bc$ for some integer $c$. ``$b$ divides $a$'' is written $b \mid a$.
And ``$b$ does not divide $a$'' is written $b \nmid a$.


\vspace*{5mm}
\textbf{Example}:( from the book) 

$3 \mid 24$ because $24=3\cdot8$, 
but $3 \nmid 17$. Negative divisors are allowed: 
$-6 \mid 54$ because $54=( -6) \cdot( -9) $, 
but $-6 \nmid -13$.
\vspace*{5mm}

\textbf{Note}: If $b$ divides $a$, then $a = bc$ for some $c$. Hence $-a= b(-c)$, so that
$b \mid (-a)$. An analogous argument shows that every divisor of $-a$ is also a divisor of $a$.
Therefore $a$ and $-a$ have the same divisors. 

\textbf{Note}: Suppose $a \neq 0$ and $b \mid a$. Then $a=bc$, so that 
$\left\lvert a\right\rvert=\left\lvert b\right\rvert \left\lvert c\right\rvert$. 
Consequently, $0 \leq \left\lvert b\right\rvert \leq \left\lvert a\right\rvert $. 
This last inequality is equivalent to $-\left\lvert a\right\rvert \leq b \leq \left\lvert a\right\rvert$.
Therefore 
\begin{itemize}
    \item every divisor of the nonzero integer $a$ is less than or equal to $\left\lvert a\right\rvert$;
    \item a nonzero integer has only finitely many divisors. 
\end{itemize}

\subsection*{Greatest Common Divisor}
\textbf{Definition}:
Let $a$ and $b$ be integers, not both $0$. The greatest common divisor ( gcd) of
$a$ and $b$ is the largest integer $d$ that divides both $a$ and $b$. In other words,
d is the gcd of $a$ and $b$ provided that

(1) $d\mid a$ and $d\mid b$;

(2) If $c\mid a$ and $c \mid b$, then $c \leq d$.

The greatest common divisor of $a$ and $b$ is usually denoted $( a, b)$.

\subsection*{Theorem 1.2}
Let $a$ and $b$ be integers, not both $0$, and let $d$ be their greatest common divisor. 
Then there exist (not necessarily unique) integers $u$ and $v$ such that
$d =au+ bv$.

\subsection*{Corollary 1.3}
Let $a$ and $b$ be integers, not both $0$, and let $d$ be a positive integer. 
Then $d$ is the greatest common divisor of $a$ and $b$ 
if and only if $d$ satisfies these conditions:

(i) $d \mid a$ and $d\mid b$;

(ii) if $c\mid a$ and $c \mid b$, then $c \mid d$.


\subsection*{Theorem 1.4}
lf $a\mid bc$ and $( a, b) = 1$, then $a\mid c$. 
\end{document}