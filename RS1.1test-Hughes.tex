\documentclass{article}
\usepackage[utf8]{inputenc}
\usepackage{amsmath}
\usepackage{graphicx}
\usepackage{BeginnerStyleFile}
\graphicspath{ {images/} }

\title{1.1 The Division Algorithm}
\author{Evan Hughes}
\date{January 2023}

\begin{document}
\maketitle

\section*{Revisiting \integers}
\integers{} is the set of all integers.  $\Z = \{ \pm 1, \pm 2, \pm 3, \ldots\}$.
\vspace*{5mm}

\textbf{Well Ordering Axiom}: Every non-empty set of integers has a smallest element.
However this is not true for all sets.  For example, $\mathbb{R}$ does not have a smallest element.
For any ratio $r$, there is always a smaller ratio $r/2$.
This also does not hold true for \integers{} because there is no smallest negative integer. 


\section*{Understanding Division}

One can start by writing out what division is verbally. It can be written as: 
\begin{center}
    dividend = (quotient) $\times$ (divisor) + (remainder)
\end{center}

\subsection*{Theorem 1.1: The Division Algorithm}
Let $a, b$ be Integers with $b > 0$. Then there exist unique Integers $q$ and $r$ such
that
\begin{center}
    $a = bq + r$  and    $0 \leq{} r < b$.
\end{center}
\textbf{Theorem 1.1} allows for the possibility of the dividend $a$ being negative.
However $r$, the remainder, is required to be positive and less than the divisor $b$.

An \textbf{example} of why the last requirement is necessary is if $a = -14$ and $b = 3$
as it leaves 3 possibilities for $q$ and $r$. If you require r to be non negative there is always one solution.

\subsection*{Proof of Theorem 1.1}
Let $a$ and $b$ be integers with $b > 0$. 
Consider the set $S$ of all integers of the form
\begin{center}
    $a - bx$, where x is an integer and $a - bx \geq{} 0$.
\end{center}

\textbf{Step 1}: Show that $S$ is non-empty.
You do this by finding a value for x such that $a-bx \geq{} 0$.
$a-bx$ is in $S$ when $x = \left\lvert -a\right\rvert $, which means $S$ is nonempty.
\medskip

\textbf{Step 2}: Find $q$ and $r$ such that $a = bq + r$ and $0 \leq{} r < b$.
By the \textbf{Well Ordering Axiom}, $S$ has a smallest element.  This smallest element is $r$.
Since $r \in S \implies r \geq 0$ and $r = a -bx$ for some $x$, like $x = q$.
Meaning $a = bq +r$ and $r \geq 0$.

\textbf{Step 3}: Show that $r < b$.
To show that $r < b$, we must show that $r \geq b$ is false. Then $r-b\geq 0$, so that
\begin{center}
    $0 \leq r-b=( a-bq) -b =  a-b( q+1)$.

    $a-b( q+1) = r-b < r$.

    $a=bq+$ and $0\leq r <b$
\end{center}

\textbf{Step 4}: Show that r and q are the only numbers with these properties.
To prove uniqueness, we suppose that there are integers
$q_1$ and $r_1$ such that $a = bq_1 + r_1$ and $0 \leq r_1 < b$, and prove that $q_1 = q$
and $r_1 = r$.
\end{document}