\documentclass{article}
\usepackage[utf8]{inputenc}
\usepackage{amsmath}
\usepackage{graphicx}
\usepackage{BeginnerStyleFile}
\graphicspath{ {images/} }

\title{Reading Summary: Appendix C, D}
\author{Evan Hughes}
\date{January 2023}

\begin{document}
\maketitle

\section*{Appendix C Well Ordering and Induction}
\subsection*{Well Ordering}
The subset of nonnegative integers will be
denoted by N. Thus 
\begin{center}
    $\mathbb{N}  = \{ 0,1,2,3,\ldots \}. $
\end{center}

\textbf{Well Ordering Axiom} \emph{Every nonempty set of } $\mathbb{N}$ \emph{contains a smallest a smallest element}
This axiom might not hold if $\mathbb{N}$ is replaced with some other set of numbers.
Well Ordering Axiom also creates the proof method known as \textbf{Induction}.
Induction can be used to prove statements like 
\begin{center}
    A set of $n$ elements has $2^n$ subsets.
\end{center}
    
Denote this statement by the symbol $P(n)$ and see that there are infinitely
many statements, one for each possible value of n: 


$P(0)\colon$ a set of 0 elements has $2^0 = 1$ subsets

$P(1)\colon$ a set of 1 elements has $2^1 = 2$ subsets

$P(2)\colon$ a set of 1 elements has $2^4 = 4$ subsets

$P(3)\colon$ a set of 1 elements has $2^3 = 8$ subsets


Then you can prove 
\begin{center}
    $P(n)$ is a true statement for $n \in \mathbb{N}$
\end{center}

\subsection*{Theorem C.1 The principle of Mathematical Induction}

Assume that for each nonnegative Integer $n$, a statement $P(n)$ is given. If
\begin{enumerate}
    \item $P(0)$ is a true statement; and
    \item Whenever $P(k)$ is true, then $P(k+1)$ is also true
\end{enumerate}
then $P(n)$ is a true statement for every $n \in \mathbb{N}$. 

To use Induction you must make sure that your set satisfies both cases. Without satisfing both Induction does not work

\textbf{Example$\colon$} Proof that $\frac{n(n+1)}{2}$ is the sum of the first $n$ nonnegative integers
First you need to prove the base case $P(0)$
\begin{equation}
    P(0) = \frac{0(0+1)}{2} = 0
\end{equation}
this holds true, the sum of the first 0 nonnegative integers is 0.

Next the inductive step, you need to prove that when $P(k)$ is true then $P(k+1)$ is also true
\begin{equation}
    P(k) = \frac{k(k+1)}{2}
\end{equation}

\begin{equation}
    P(k+1) = \frac{(k+1)((k+1)+1)}{2} = \frac{(k+1)(k+2)}{2}
\end{equation}

then you need to prove that that is equal to $P(k) + (k+1)$

\begin{equation}
    \frac{k(k+1)}{2} + (k+1) = \frac{(k+1)(k+2)}{2} = P(k+1)
\end{equation}
this completes the induction because you have proven the when $P(k)$ is true then $P(k+1)$ is true. $\blacksquare$

\subsection*{Theorem C.2 The principle of Complete Induction}

\begin{enumerate}
    \item $P(0)$ is a true statement; and
    \item Whenever $P(j)$ is true for all $j$ such that $0 \leq j < t$, then $P(t)$ is also true
\end{enumerate}
then $P(n)$ is a true statement for every $n \in \mathbb{N}$.

\textbf{Example$\colon$}(from the book) Proof of Theorem C.2

For each $n \in \mathbb{N}$, let $Q(n)$ be the statement
\begin{center}
$P(J)$ is true for all $j$ such that $O \leq j \leq n$.
\end{center}
We will use Theorem C.1 to prove that $Q(n)$ is true for every $n \in \mathbb{N}$.
Now $Q(O)$ is the statement
\begin{center}
    $P(J)$ is true for all $j$ such that $0 \leq j \leq 0$. 
\end{center}

This says, $Q(O)$ is just the statement ``$P(O)$ is true''. Assume that Q(k) is
true, that is, 
\begin{center}
    $P(J)$ is true for all $j$ such that $O \leq j \leq k$.
\end{center}

By hypothesis (2) (with $t = k + 1$), we conclude the $P(k + 1)$ is also true.
Therefore, $P(J)$ is true for all $j$ such that $O \leq j \leq k + 1$, that is, $Q(k + 1)$
is a true statement. Thus we have shown that whenever $Q(k)$ is true, then
$Q(k + 1)$ is also true. By the Principle of Mathematical Induction, $Q(n)$
is true for every $n \in \mathbb{N}$, and the proof is complete. $\blacksquare$  

\subsection*{Theorem C.3}

Let $r$ be a positive integer and assume that for each $n \geq r$ a statement $P(n)$
1s given. If
(i) $P(r)$ is a true statement;
and either
(ii) Whenever $k \geq r$ and $P(k)$ is true, then $P(k + 1)$ is true;
or
(ii') Whenever $P(j)$ is true for all $j$ such that $r \leq j < t$, then $P(t)$ is true,
then $P(n)$ is true for every $n \geq r$. 

\subsection*{Theorem C.4}
The following are equivalent
\begin{enumerate}
    \item The Well Ordering Axiom
    \item The principle of Mathematical Induction
    \item The principle of Complete Induction
\end{enumerate}

\section*{Appendix D Equivalence Relations}

If $A$ is a set, then any subset of $A * A$ is called a relation on $A$. $A$ relation $T$ on $A$
is called an \textbf{equivalence relation} provided that the subset $T$ is

(i) \textbf{Reflexive}: $a ~ a \in T$ for every $a \in A$.

(ii) \textbf{Symmetric}: If $a ~ b \in T$, then $b ~ a~$.

(iii) \textbf{Transitive}: If $a ~ b \in T$ and $b ~ c  \in T$, then $a ~ c$. 

\end{document}