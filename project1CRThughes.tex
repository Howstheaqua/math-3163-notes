\documentclass[12pt]{article}
\usepackage[utf8]{inputenc}
\usepackage{amsmath}
\usepackage{graphicx}
\usepackage{scrextend}
\usepackage{hyperref}
\usepackage{BeginnerStyleFile}
\usepackage[explicit]{titlesec}
\graphicspath{ {images/} }
\linespread{1.5}
\title{Project 1: The Chinese Remainder Theorem}
\author{Evan Hughes}
\date{March 2023}

\begin{document}
\maketitle
\begin{abstract}
The Chinese Remainder Theorem (CRT) is an ancient result 
in number theory that provides a method to solve systems 
of linear congruences. It was developed as a means to solve 
practical problems related to modular arithmetic. In 
ancient China, the theorem was used to solve problems 
in calendar calculations, land division, and taxation. 
The theorem's usefulness extends beyond these practical 
applications, and it has become a foundational result in 
number theory, algebra, cryptography, and computer science.

This research paper provides an overview of the history 
and development of the Chinese Remainder Theorem, from 
its ancient origins to its modern formulations and 
applications. We explore the motivation behind the theorem's 
development, its applications in ancient China, and its 
significance in mathematics and computer science. We also 
investigate various extensions and generalizations of the 
theorem and discuss recent advances in the algorithmic and 
computational aspects of the theorem.
\end{abstract}

\pagebreak
\section*{Introduction}
The Chinese Remainder Theorem (CRT) is a mathematical result that provides 
a method to solve a system of linear congruences with pairwise relatively 
prime moduli. The theorem's origin can be traced back to ancient China, where 
it was developed as a solution to practical problems in modular arithmetic. 
The Chinese mathematical treatise Sunzi Suanjing, written in the 3rd century 
AD, contains the earliest known reference to the theorem. The theorem was 
further developed by the Chinese mathematician Qin Jiushao during the Song 
Dynasty (960-1279). Qin Jiushao's works, such as Mathematical Treatise in 
Nine Sections, demonstrated the theorem's applications in the fields of 
astronomy, calendar making, and music theory. The theorem's usefulness in 
these practical applications made it a fundamental result in ancient Chinese 
mathematics.

The theorem was further developed by the Ming Dynasty mathematician 
Zhu Shijie in his book Jade Mirror of Four Unknowns. Zhu Shijie provided 
a more general formulation of the theorem that extended its applications 
to solving systems of simultaneous congruences. This work established the 
CRT as a foundational result in Chinese mathematics and set the stage for 
its later development in Western mathematics.

The Chinese Remainder Theorem's ancient origins highlight the importance of 
mathematical discoveries in solving practical problems and advancing 
scientific knowledge. The Chinese calendar, for instance, was based on a 
60-year cycle, and its computation required keeping track of both the lunar 
and solar cycles. The overlap between the cycles made it difficult to 
perform calculations. The CRT provided a method to divide a year into 
different components, each of which could be computed separately and 
then combined to determine the calendar date. The CRT was also used in 
land division, where it was essential to ensure that each person received 
an equal share of the land. The CRT was used to calculate the size of each 
plot so that each person received an equal share of the land. These 
practical applications demonstrate the theorem's utility and demonstrate 
how mathematical discoveries can be applied to solve real-world problems. 
The theorem's development in ancient China paved the way for its later 
applications in modern mathematics and computer science, demonstrating 
the timelessness of mathematical discoveries and their impact on our world.


\section*{Proof of the Chinese Remainder Theorem (CRT)}

In order to prove the Chinese Remainder Theorem, we must prove 4 items 
with the following definitions:

\begin{addmargin}[1cm]{1cm}
    For each $i = 1, 2, \ldots, r$, $let$ $N_i$ be the 
    product of all the moduli $m_j$ for $j \neq i, N_i = m_1 \cdots 
    m_{i-1}m_{i+1}\cdots m_r$.
\end{addmargin}

The items we need to prove are as follows:
\begin{enumerate}
    \item For each $i$, show that gcd ($N_i, m_i$) = 1, and that there are integers $u_i$ and $v_i$
    \item For each $i$, and $j$ with $i\neq j$, show that $N_i u_i = 0 \mod m_j$
    \item For each $i$, show that $N_i u_i = 1 \mod m_i$
    \item Find a $\mathbb{Z}$-linear combination $x$ of the $N_i u_i$ which solves the entire system of linear congruences,
    \begin{center}
        $x = a_1 (\mod m_1)$

        $x = a_2 (\mod m_2)$

        $\vdots$

        $x = a_r (\mod m_r)$
    \end{center}
\end{enumerate}

To start we will prove $(1)$, For each $i$, show that $\gcd (N_i, m_i) = 1$, 
and that there are integers $u_i$ and $v_i$ such that $N_i u_i + m_i v_i = 1$.
First to show that the $\gcd (N_i, m_i) = 1$ we can show that the $\gcd$ of $m_i$ and 
each element of $N_i$ is 1. This is because $N_i$ is the product of all the
moduli $m_j$ for $j \neq i$, and so each element of $N_i$ is relatively prime to $m_i$.
Because $\gcd(a,b) = \gcd(c,b) = 1 \implies \gcd(ac,b) = 1$ means that the $\gcd$ of $N_i$ and $m_i$ is 1.
Now to show that there are integers $u_i$ and $v_i$ such that $N_i u_i + m_i v_i = 1$ we can use the already
proven fact that $\gcd (N_i, m_i) = 1$ to show that $N_i u_i + m_i v_i = 1$ by using Bézout's lemma.

Now we will prove $(2)$, For each $i$, and $j$ with $i\neq j$, 
show that $N_i u_i = 0 \mod m_j$.
To show this we can use the fact that $N_i$ is equal to the product of all 
the moduli $m_j$ for $j \neq i$. This means that $N_i u_i$ is always a multiple
of $m_j$ for $j \neq i$. This means that $N_i u_i = 0 (\mod m_j)$.

Now we will prove $(3)$, For each $i$, show that $N_i u_i = 1 \mod m_i$.
In order for $N_i u_i = 1 (\mod m_i)$ then $N_i u_i = k m_i + 1$ 
for any integer $k$. Using the proof of $(1)$ we know that $N_i u_i + m_i v_i = 1$.
Which means that $N_i u_i = 1 - m_i v_i$. Through substitution we can see that
$1 - m_i v_i = k m_i + 1$. It follows that $k m_i = -v_i m_i$. Because $k$ can
be any integer and $k m_i = -v_i m_i$ then $k = -v_i$. This proves that for all $i$
$N_i u_i = 1 (\mod m_i)$.


Now we will prove $(4)$, Find a $\mathbb{Z}$-linear combination $x$ of 
the $N_i u_i$ which solves the entire system of linear congruences.
To find a $\mathbb{Z}$-linear combination $x$ of the $N_i u_i$ which solves the entire system of linear congruences
we can use the fact that $N_i u_i = 1 (\mod m_i)$ to show that $a_1(N_1 u_1) = a_1 (\mod m_1)$, $a_2(N_2 u_2) = a_2 (\mod m_2)$, $\ldots$, $a_r(N_r u_r) = a_r (\mod m_r)$.
Now we can add all of these elements together to get $x = a_1(N_1 u_1) + a_2(N_2 u_2) + \ldots + a_r(N_r u_r)$.
$x = a_y (\mod m_y)$ for all $y$ because in a given $(\mod m_y)$ 
the only part that gives a value is $a_y(N_y u_y)$ and the rest of the linear combination
, $a_1(N_1 u_1) + \cdots + a_{y-1}(N_{y-1} u_{y-1}) + a_{y+1}(N_{y+1} u_{y+1}) + \cdots a_r(N_r u_r) = 0$ as proven in $(2)$ and $(3)$.
So this $\mathbb{Z}$-linear combination $x$ of $N_i u_i$
solves the entire system of linear congruences, 
\begin{center}
    $x = a_1 (\mod m_1)$

    $x = a_2 (\mod m_2)$

    $\vdots$

    $x = a_r (\mod m_r)$
\end{center}

\section*{Example Problems}

\subsection*{Use the CRT to solve the linear congruence $17x = 9 (\mod 276)$}

Using the CRT $17x = 9 (\mod 276)$ we can find $x$ by using the following steps:
First we need to break up this linear congruence into multiple coprime linear congruences.
\begin{center}
    $17x = 9 (\mod 3)$

    $17x = 9 (\mod 4)$

    $17x = 9 (\mod 23)$
\end{center}
This can be reduced to the following system of linear congruences:
\begin{center}
    $x = 0 (\mod 3)$

    $x = 1 (\mod 4)$

    $x = 13 (\mod 23)$
\end{center}
Using the CRT this gives that $x = 105 (\mod 276)$.

\subsection*{A gang of} 17 bandits stole a chest of gold coins. When they tried
to divide the coins equally among themselves, there were three left over. This caused a fight in
which one bandit was killed. When the remaining bandits tried to divide the coins again, there
were ten left over. Another fight started, and five of the bandits were killed. When the survivors
divided the coins, there were four left over. Another fight ensued in which four bandits were killed.
The survivors then divided the coins equally among themselves, with none left over. What is the
smallest possible number of coins in the chest?

To write this out as a number problem gives

\begin{center}
    $x = 3 (\mod 17)$

    $x = 10 (\mod 16)$

    $x = 4 (\mod 11)$

    $x = 0 (\mod 7)$
\end{center}
Using the CRT we know that $x$ has a solution. We will start by solving the first 2 congruences then the last 2.

From the CRT we get that given 2 congruences, $x = a (\mod m)$ and $x = b (\mod n)$, a solution is given by $t = bmu + anv$.

With the first two congruences we get that $m = 17, n = 16, a = 3, b = 10$.

We also know that $17u + 16v = 1; u = 1, v = -1$.

Plugging these into the previous equation gives $t = 10(1)(17) + 3(-1)(16) = 122$.

So these two congruences give $x = 122 (\mod 17*16)$.

Now the next two congruences give $m = 11, n = 7, a = 4, b = 0$ and $11u + 7v = 1; u = 2, v = -3$.

Plugging these into the previous equation gives $t = 0(2)(11) + 4(-3)(7) = -84$.

These two congruences give $x = -84 (\mod 11*7)$.

This leaves us with $x = 122 (\mod 272)$ and $x = 70 (\mod 77)$.

To reduce this one we get $m = 272, n = 77, a = 122, b = 70$ and $272u + 77v = 1; u = -15, v = 53$.

Plugging these into the previous equation gives $t = 70(-15)(272) + 122(53)(77) = 212282$.

This gives us $x = 212282 (\mod 272*77)$ which reduces to $x = 2842 (\mod 20944)$.

This means the smallest possible number of coins in the chest is 2842.


\subsection*{Let $t_n$ be the $n$th trianglular number}
defined as $t_n = 1 + 2 + \cdots + n$. For which values of n does $t_n$ divide $t^2_1 + t^2_2 + \cdots + t^2_n$?
We know that $t^2_1 + t^2_2 + \cdots + t^2_n$ is equal $t_n\frac{(3n^3+12n^2+13n+2)}{30}$.
So to find when $t_n$ divides $t^2_1 + t^2_2 + \cdots + t^2_n$ we need to find when $\frac{(3n^3+12n^2+13n+2)}{30}$
is an integer, because that will give us $t_n$ times an integer.

So we need to solve $(3n^3+12n^2+13n+2) = 0 (\mod 30)$. $30$ can factor into $2*3*5$ so we can break this up into 3 linear congruences.

To start with $(\mod 2)$ we get $n^3 + n = 0 (\mod 2)$, which is true for all integers $n$.

In $(\mod 3)$ we get $n + 2 = 0 (\mod 3)$, which is only true if $n = 1 (\mod 3)$.

Next in $(\mod 5)$ we get $3n^3 - 3n^2 + 3n - 3 = 0 (\mod 5)$, which factors to
$(n-1)(n^2+1) = 0 (\mod 5)$, which means $n = 1,2$, or $3 (\mod 5)$.

This leaves us with the conditions that $n = 1 (\mod 3)$ and $n = 1,2,3 (\mod 5)$.

To solve these congruences we can use the CRT.

To take the first 2 congruences we get $m = 3, n = 5, a = 1, b = 1$ and $3u + 5v = 1; u = 2, v = -1$.

This gives $t = 1(2)(3) + 1(-1)(5) = 1$. Which means $n = 1 (\mod 15)$.

Next we will use the first and third congruences to get $m = 3, n = 5, a = 1, b = 2$ and $3u + 5v = 1; u = 2, v = -1$.

this gives $t = 2(2)(3) + 1(-1)(5) = 7$. Which means $n = 7 (\mod 15)$.

Finally using the first and last congruences to get $m = 3, n = 5, a = 1, b = 3$ and $3u + 5v = 1; u = 2, v = -1$.

This gives $t = 3(2)(3) + 1(-1)(5) = 13$. Which means $n = 13 (\mod 15)$.

So the solutions for when $t_n$ divides $t^2_1 + t^2_2 + \cdots + t^2_n$ is when $n = 1, 7, 13 (\mod 15)$.


\section*{Generalization to non-coprime moduli}
To generalize the CRT to non-coprime moduli we can start by letting
$m,n,a,b$ be any integers, and $g = \gcd(m,n)$; $M = lcm(m,n)$ and consider
the following system of linear congruences:
\begin{center}
    $x = a (\mod m)$

    $x = b (\mod n)$
\end{center}
If $a = b ( \mod g)$ then this system has a unique solution modulo $M = mn/g$. Otherwise, no solution.
If we use Bezout's Lemma to write $g = um + vn$ then the solution is given by $x = \frac{avn+bum}{g}$

\section*{Conclusion}

In conclusion, the Chinese Remainder Theorem has a rich history that dates 
back to ancient China. It was developed to solve practical problems in 
modular arithmetic, such as calendar making, land division, and music theory. 
The theorem's development in ancient China paved the way for its later 
applications in modern mathematics and computer science, demonstrating the 
timelessness of mathematical discoveries and their impact on our world. The 
CRT has since become a fundamental result in number theory, algebra, and 
cryptography.

Modern applications of the CRT include its use in coding theory, computer 
science, and physics. In coding theory, the CRT is used in the construction 
of error-correcting codes, which are essential for transmitting information 
over noisy channels. In computer science, the CRT provides a basis for many 
secure communication protocols, such as RSA encryption. In physics, the CRT 
is used in the study of the quantum mechanics of spin systems and in the 
computation of energy levels in complex systems.


\pagebreak
\section*{Sources}
\begin{itemize}
    \item \href{https://kconrad.math.uconn.edu/blurbs/ugradnumthy/crt.pdf}{https://kconrad.math.uconn.edu/blurbs/ugradnumthy/crt.pdf}
    \item \href{https://people.math.harvard.edu/~knill/crt/lib/Kangsheng.pdf}{https://people.math.harvard.edu/~knill/crt/lib/Kangsheng.pdf}
    \item \href{https://mathoverflow.net/questions/10014/applications-of-the-chinese-remainder-theorem}{https://mathoverflow.net/questions/10014/applications-of-the-chinese-remainder-theorem}
\end{itemize}




\end{document}