\documentclass{article}
\usepackage[utf8]{inputenc}
\usepackage{amsmath}
\usepackage{graphicx}
\usepackage{BeginnerStyleFile}
\usepackage{scrextend}
\usepackage{ifthen}
\usepackage{listings}
\graphicspath{ {images/} }

\newcommand{\gfcb}[1]{%
    \fcolorbox{white}{gray!10!}{\quad\strut #1\quad}
    }
\newcommand{\cop}[1]{%
    \texttt{\detokenize{#1}}
    }

\title{Homework 2}
\author{Evan Hughes}
\date{January 2023}

\begin{document}
\maketitle
\begin{enumerate}

    %%%%%%%%%%%%%%%%%%%%
    \item Let $R$ be a ring and fix $b \in R$.
    %%%%%%%%%%%%%%%%%%%%
    
    \begin{enumerate}
        \item Show that $S = {n1_R \in R : n \in \mathbb{Z}}$ is a subring of $R$.
        \item Show that $T = {rb \in R : r \in \mathbb{Z}}$ is a subring of $R$.
        \item Is $M = {nb \in R : n \in \mathbb{Z}}$ always a subring of R? If no, give a counterexample. If yes, prove it.
    \end{enumerate}
    
    %%%%%%%%%%%%%%%%%%%%
    \item Let $R$ be a ring.
    %%%%%%%%%%%%%%%%%%%%
    \begin{enumerate}
        \item Show that $R$ has exactly on zero, $0_R$.
        \item If $R$ also has an identity, $1_R$, show that $1_R$ is the only identity.
        \item Can a unit in $R$ have more than one inverse? If yes, give an example. If no, prove it.
    \end{enumberate}

    %%%%%%%%%%%%%%%%%%%%
    \item Let $R$ be a ring and $S, T$ be subrings of $R$. In each of the following cases, provide a proof if the answer is yes, and provide a counterexample if the answer is no.
    %%%%%%%%%%%%%%%%%%%%
    \begin{enumerate}
        \item Is $S \cap T$ a subring of $R$?
        \item Is $S \cup T$ a subring of $R$?
        \item If either $S \cap T$ or $S \cup T$ is not a subring of $R$, can they be extended to a subring of $R$? If so, explain how.
        \item Let $I$ be an arbitrary index set and ${S_i \subseteq R:\forall i \in I, S_i \leq R}$ an $I$-indexed set of subring of $R$. Is $\cap{S_i: i \in I}$ a subring
    \end{enumerate}
\end{enumerate}
    
\end{document}