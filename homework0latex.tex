\documentclass{article}
\usepackage[utf8]{inputenc}
\usepackage{amsmath}
\usepackage{graphicx}
\usepackage{BeginnerStyleFile}
\graphicspath{ {images/} }

\title{Homework 0}
\author{Evan Hughes}
\date{January 2023}

\begin{document}

\maketitle

\subsection*{1. Fill in the following table}
\begin{tabular}{||c c c c c c c c||}
    $p$ & $q$ & $\lnot p$ & $\lnot q$ & $p \implies q$ & $\lnot p \implies \lnot q$ & $\lnot q \implies \lnot p$ & $\lnot ( p \Longleftrightarrow  q) $ \\
    \hline
    T & T & F & F & T & T & T & F \\
    T & F & F & T & F & T & F & T \\
    F & T & T & F & T & F & T & F \\
    F & F & T & T & T & T & T & T \\

\end{tabular}

\subsection*{2. Which of the following expressions are tautologies?}

\begin{itemize}
    \item $( ( ( p \implies q) \implies p) \implies p) $ : Tautology
    \item $( ( ( p \lor  q) \land \lnot q ) \implies q)$ : Tautology
    \item $( p \implies ( \lnot p \land q ) ) $ : Not a Tautology
    \item $( ( ( p \lor \lnot q) \implies r) \implies q)$ : Not a Tautology
    \item $( ( ( p \implies q) \lor ( r \implies s ) ) \implies ( ( p \lor r ) \implies ( q \land s) ) ) $ : Tautology
    \item $( ( ( p \implies q) \land ( r \implies s ) ) \implies ( ( p \lor r ) \implies ( q \land s) ) ) $ : Tautology
\end{itemize}

\subsection*{3. State whether the given biconditional is true or false under the assumption that all variables
are quantified over $\mathbb{R}$ . Give a brief explanation for each.}

\textbf{(a) } $x^2 = 9$ if and only if $x = 3$ : False because $ ( -3) ^2 = 9$

\textbf{(b) } $x$ is a positive number if and only if $x > 0$ : True because positive numbers are numbers that are greater than 0

\textbf{(c) } $\left\lvert x\right\rvert $ is a positive number if and only if $x \neq  0$ : True because the absolute value of a number is always positive unless its 0

\textbf{(d) } A number $x$ is rational if and only if it has a terminating decimal expansion : False because $\frac{1}{3}$ is rational but does not have a terminating decimal expansion

\textbf{(e) } Today is March 1 if and only if yesterday was February 28 : False because today is March 1st and yesterday was February 29th if its a leap year

\subsection*{\LaTeX{} Exercises}
\begin{enumerate}
    \item Please type me! Sphinx of black quartz, judge my vow!
    \item $e^{i\pi} + 1 = 0$
    \item $e^{i\Theta} = \cos(\Theta) + i\sin(\Theta)$
    \item $G_{\mu{}v}+\Lambda{}g_{\mu{}v} = \frac{8\pi{}G}{c^4}T_{\mu{}v}$ 
    \item $x = \frac{-b\pm{}\sqrt{b^2+4ac} }{2a}$
    \item $\overrightarrow{L}=\overrightarrow{r}\times \overrightarrow{t}$
    \item $( x+y ) ^n=\sum_{r = 0}^{n}( \genfrac{}{}{0px}{}{n}{r}) x^r y^{n-r}$
    \item $\sqrt{\frac{a_1^2+\dots+a_n^2}{n}} \geq \frac{a_1+\cdots+a_n}{n} \geq \sqrt[n]{a_1+\cdots+a_n} \geq \frac{n}{\frac{1}{a_1}+\dots+\frac{1}{a_n}}$
    \item $ \left\lvert \langle x , y\rangle \right\rvert \leq \langle x,x\rangle * \langle y,y\rangle $
    \item 
\end{enumerate}
A1: $\varphi \longrightarrow  ( \psi \rightarrow \varphi) $

A2: $( \varphi \longrightarrow  ( \psi \rightarrow \theta) ) \longrightarrow ( ( \varphi \rightarrow \psi) \longrightarrow ( \varphi \rightarrow \theta ) )$

A3: $( \lnot \varphi \rightarrow \lnot \psi) \longrightarrow ( \psi \rightarrow \varphi) $

\end{document}